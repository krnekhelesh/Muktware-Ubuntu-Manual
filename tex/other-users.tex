This chapter is dedicated specifically to Windows and Mac users who are thinking of switching to Ubuntu. This chapter is intended to clear some myths and provide facts that you should consider before making this switch. 

\subsection*{1. An entirely different ecosystem}
When you switch from Windows or Mac OS to Ubuntu, remember that you are not just switching to another operating system but rather also an entire ecosystem of applications that was built for that operating system. In that sense, you were not using Internet Explorer, but rather \emph{Internet Explorer for Windows}. So do not be startled by the fact that Internet Explorer does not run in Ubuntu. Applications in Windows or Mac OS have been encouraged to be platform specific meaning it will only work in that specific operating system. This essentially locks you to Windows or Mac OS without so much of a choice. Use this opportunity to try out other open source applications such as Firefox, Libreoffice which are cross-platform, meaning they run on all operating systems such as Windows, Mac OS, Ubuntu (Linux). If you want to switch to Ubuntu, you need to break free of this dependency on Windows applications to use Ubuntu. The rewards of this are enormous as there are tons of applications out there.

\subsection*{2. The Anthropic principle}
By definition, everything you currently do in Windows works in Windows, otherwise you wouldn't have been able to do it. That doesn't mean that Windows does everything you might want it to do. In fact, if you really paid attention, you'd be aware of all of the things you wanted to do in Windows but couldn't. So the setup you have now is not necessarily the setup you want, it's just the one you settled for.  Since you are trying a switch to a different platform, now is the perfect time to question whether or not it's really what you want, or if you can do something better. Don't get stuck trying to re-create a solution you  used in Windows to a problem that doesn't actually exist on Linux. You have a whole world of new possibilities in front of you now, take advantage of that and question your old habits. For instance, Windows updates which roll out every now and then only include updates to the operating systems and other applications created by Microsoft like Windows Media Player. All the other applications installed by you are not updated and hence not up to date. This might be acceptable but a more desirable solution would be to update all the applications on your system in just one go. 

\subsection*{3. Look before you install}
If you look at typical fresh install of Windows for instance, you are presented with the bare necessary software pre-installed. Windows by default does not come with an office suite, Winzip utility etc. Because of this, Windows users generally have a habit of approaching a new installation with the desire to install more stuff. Ubuntu is different, and comes with a very large selection of useful applications already installed. So before you go off looking for an installer for Office, AIM or WinZip, look at what you already have because there is a good chance something is already there (LibreOffice, Empathy, File Roller).

\subsection*{4. The need to learn to use Ubuntu}
Accept the fact that Ubuntu is different from other operating systems. This does not essentially mean that you have to relearn every aspect of using your computer from scratch. It is true that you may have to relearn certain aspects, however this is entirely normal. This is something which can be seen everywhere. Imagine a Windows user trying to use Mac OS for the very first time. It will take time before you get used to the various touch gestures, the way of installing applications in Mac OS and file structure etc. The same applies when you switch to Ubuntu.  That said, this sole purpose of this manual is help you in this area. 
