%appendix B
\index{Terminal commands}
\begin{table}[h]
	\centering
	\begin{tabular}{|l|l|p{11cm}|}
	\hline
	\multicolumn{3}{|c|}{\textbf{Basic commands}} \\
 	\hline
 	cd & directory\_name & Navigate in/out of a  directory\\
 	& .. & cd with two dots enables you to go back one step. For example you are at /home/DIR1/DIR2. If you type cd .. you will be returned to DIR1. \\
	& & cd without any arguments returns you straight back to the root directory (that is usually home directory). \\
    \hline
    clear & & Clear the terminal of all output \\
    \hline
    
    cp & file directory & Copy files  to some directory \\
    & -r  directory\_name & -r argument together with cp enables you to copy some directory to some other directory. \\
	\hline
	help & & Lists commands used under terminal \\
	\hline
	ls & & List all the visible files and folders that are under the current directory \\        
 	& -al & List all files and folders including hidden files\\
 	\hline
 	mkdir & directory\_name & Create a new directory \\
 	\hline
 	mv & file\_name directory & Enables you to move a file or a directory to some other directory. It works with same principle like cut command under GUI.  \\
 	\hline
 	man & command\_name & If you type man ls for example you will get a detailed explanation about ls command. This goes for any other command. Man is actually terminal's main help system. \\
 	\hline
 	pwd & & Display name of current/working directory \\
	\hline
	rm & file\_name & Removes only files \\
	& -R  directory\_name & Command rm with -R enables you to erase entire directory together with its content. Command rm with -R enables you to erase entire directory together with its content. \\
	\hline
	rmdir & & With rmdir command you can erase only empty directories \\
	\hline
	sudo & & This command stands in front some other command which can be only executed by administrator or super user. When cruical operation is needed to be executed you have to use sudo. Otherwise it will not work. \\
	
	\hline
	\multicolumn{3}{|c|}{\textbf{Administrative commands}} \\
 	\hline
 	sudo apt-get & install package\_name  & You can install a new package or a tool that you specify. \\
 	& update & Update your system's repository (database).  \\
 	& dist-upgrade & Download packages for your system from the recently updated repository. \\
 	& autoremove & Remove unwanted packages automatically.\\
 	& autoclean & Removes all .deb files for packages that are no longer installed on your system.\\
 	& clean & Removes all packages from the package cache. Note: autoclean,clean, autoremove help you to free up disk space from unnecessary packages.\\
   \hline
	
	\hline
	\end{tabular}
	\label{tab:terminal-commands}
	\caption{Terminal commands }
\end{table}

\par \noindent \textbf{Tip}: Use up arrow  on your keyboard. If you press up arrow on your keyboard you will not have to type some previously typed commands. Terminal actually remembers what you have typed recently. This is a time saviour. 